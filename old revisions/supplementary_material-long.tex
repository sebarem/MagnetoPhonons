
%\documentclass[aip,preprint,graphicx]{revtex4-1}
\documentclass[pra,aps,superscriptaddress,preprint]{revtex4-1}
\usepackage{graphicx}
\usepackage{amsmath}
\usepackage{amssymb}


\renewcommand{\thepage}{S\arabic{page}}  
\renewcommand{\thesection}{S\arabic{section}}   
\renewcommand{\thetable}{S\arabic{table}}   
\renewcommand{\thefigure}{S\arabic{figure}}
\renewcommand{\theequation}{S\arabic{equation}}

\begin{document}

\title{Electrostatic control of magnetophonons in Graphene} %Title of paper

\author{Sebastian R\'{e}mi}
\affiliation{Boston University, Department of Physics, 590 Commonwealth Ave, Boston, MA 02215}

\author{Bennett B. Goldberg}

\affiliation{Boston University, Department of Physics, 590 Commonwealth Ave, Boston, MA 02215}
\affiliation{Boston University, Department of Electrical and Computer Engineering}
\affiliation{Boston University, Photonics Center, 8 St. Mary's St, Boston, MA 02215}

\author{Anna K. Swan}
 \email[]{swan@bu.edu}
\affiliation{Boston University, Department of Electrical and Computer Engineering}
\affiliation{Boston University, Department of Physics, 590 Commonwealth Ave, Boston, MA 02215}
\affiliation{Boston University, Photonics Center, 8 St. Mary's St, Boston, MA 02215}


\date{\today}



\pacs{}% insert suggested PACS numbers in braces on next line

\maketitle %\maketitle must follow title, authors, abstract and \pacs

% Body of paper goes here. Use proper sectioning commands. 
% References should be done using the \cite, \ref, and \label commands
\section{Sample Preparation}
Field effect samples were prepared by standard micro fabrication processes. We produce single layer graphene by scotch tape exfoliation from cystals of natural graphite. We exfoliate on degenerately doped Si (n-type, $\rho < 0.005\; \Omega cm$) wafers covered with thermally grown SiO$_2$ of thickness $d_{SiO_2}=300\; nm$. We identify single layer graphene samples by measurements of optical contrast and Raman measurements. Next we spin a double layer Poly(methyl methacrylate). To facilitate liftoff we spin a bilayer of PMMA495 followed by PMMA950. Afterwards we define contacts on top of graphene using standard electron beam lithography and evaporate Cr/Au contacts. 
Once the sample was in vacuum in the cryostat we performed a series of in situ current annealing steps in order to clean the sample from fabrication residues. The maximum current applied was $I=1.5\; mA$ for $\sim 4\;h$.

\section{Fitting of spectra at B=0T}
We initially extract the position and width of the G-band by fitting the raw data to a single Lorentzian peak (Fig.\,\ref{b0-all-fits}(a)). We find that all spectra have a pronounced low energy shoulder, so fitting with a single peak does not always describe the spectra well. In the vicinity of $V_{bg}=0\; V$, the G-band overlaps with this peak and spectra still appear as a single Lorentzian. For larger value of $V_{bg}$, strong deviations from the single Lorentzian lineshape occur, and the position and width of the G-band are not described accurately. We find that the low energy shoulder seems to be of equal size and position for all values of $V_{bg}$, and might be caused by a spectral ghost created by the grating of our spectrometer. 

\begin{figure}
   \includegraphics[width=0.9\columnwidth]{suppl_images/B0-all-fits}
   \caption{\label{b0-all-fits}Fitting of raw spectral data (blue dots) by single Lorentzian peak (a) or two Lorentzian peaks (b) where the low energy peak is kept fixed in position, width and amplitude. Spectra are taken at different gate voltages}
\end{figure}

To achieve a better fit, we use two Lorentzian peaks where the low energy peak is kept fixed in position, width and amplitude. This model results in very acceptable fits to the data (Fig.\,\ref{b0-all-fits}(b))   

For evaluation of our measurement results as a function of the Fermi energy $E_F(V_{bg})$ we first calculated the charge carrier density $\widetilde{n}$ in the standard way from a simple capacitive model \cite{novoselov2004electric}: 

\noindent
\begin{equation}
\label{CalculationDensity}
\widetilde{n}=C\frac{V_{bg}}{e}=\frac{\varepsilon_{\mathrm{SiO}_2}\varepsilon_0}{d_{SiO_2}}\frac{V_{bg}}{e}\approx 7.18\times10^{10}\frac{cm^{-2}}{V}V_{bg}
\end{equation}

\noindent
with the dielectric constant of SiO$_2$, $\varepsilon_{\mathrm{SiO}_2}=3.9$ and thickness of SiO$_2$, $d_{SiO_2}=300\; nm$. $\varepsilon_0$ is the permittivity of free space. The Fermi energy is then calculated as:

\noindent
\begin{equation}
\label{CalculationFermiEnergy}
E_F = \hbar v_F \sqrt{\pi n}
\end{equation}

\noindent
In Fig.\,\ref{b0-combi} we show the fitresults as a function of $E_F$. We used $v_F = 1.10 \times 10^6 \; ms^{-1}$ here. We discuss the choice of $v_F$ in section \ref{Quant-Descr}.
\noindent
\begin{figure}
   \includegraphics[width=0.9\columnwidth]{suppl_images/B0-combi}
   \caption{\label{b0-combi}Combined fitresults for single and double Lorentzian fits. Black lines are results of single Lorentzian fit, grey lines are results from double Lorentzian fit and red dots mark the selected data points for further analysis }
\end{figure}
\noindent
Black lines are the result from single Lorentzian fits and grey from double Lorentzian fits. We observe a local dip in the G-band position around $E_F=\hbar\omega_G/2$ (grey vertical lines) which is especially visible in the data extracted from double Lorentzian fit results.

A logarithmic divergence at $E_F=\hbar\omega_G/2$ softening the phonons, has been predicted by Ando \cite{tsuneya2006anomaly}. 
To our knowledge this effect has never been observed in experiments on single layer graphene most likely due to charge inhomogeneity and electron-hole puddles observed in most samples \cite{yan2007electric,martin2007observation}. 

The linewidth is unusually large as measured by double Lorentzian fits when compared to other experiments \citep{yan2007electric}. We therefore choose to use data from single Lorentzian fits for $E_F<\left|\hbar\omega_G\right|$, and we use data from double Lorentzian fits for all other regions. The data choice is illustrated in Fig. \ref{b0-combi} where we marked the data points selected for further analysis by red dots. 

\newpage
\section{Ando-model of the phonon anomaly in graphene}

We use a model proposed by Ando \cite{tsuneya2006anomaly} to describe the gatesweep behavior at $B=0\; T$. Due to a Kohn anomaly at the $\Gamma$ point of the phonon dispersion relation, the behavior of the G-band deviates from that of typical metals. The anomalous behavior is explained by the breakdown of the Born-Oppenheimer approximation \cite{pisana2007breakdown}, i.e. the assumption that the electrons follow the lattice motion instantaneously is not valid in graphene. Following \cite{tsuneya2006anomaly}, the shift of the G-band at $B=0\; T$ is described by the self-energy of the optical phonon

\noindent
\begin{equation}
\label{ando-anomaly}
\Pi(\varepsilon_0) = \lambda \varepsilon_F -\frac{\lambda}{4}\left(\varepsilon_0+i\delta\right)\left(\mathrm{ln}\left(\frac{\varepsilon_0+2\varepsilon_F+i\delta}{\varepsilon_0-2\varepsilon_F+i\delta}\right)+i\pi\right)
\end{equation}

\noindent
Eqn.\,(\ref{ando-anomaly}) is obtained in the $B=0\; T$ limit of the expressions in the main paper. $\lambda$ is the dimensionless electron-phonon coupling, $\delta$ is a phenomenologic broadening factor describing electronic level broadening ,and $\epsilon_0$ is the unperturbed optical phonon energy. The shift and broadening are then given by the real part and imaginary part of Eqn.\,(\ref{ando-anomaly}).

\noindent
\begin{eqnarray}
\label{ando-shift}
\varepsilon_G &=& \varepsilon_0 + \mathrm{Re}\Pi(\varepsilon_0) \\
\Gamma_G &=& \delta_0 + \mathrm{Im}\Pi(\varepsilon_0)
\end{eqnarray}

\noindent
In Fig.\,\ref{ando-model} we show numerical results calculated for $\lambda = 4.8\times 10^{-3}$ and $\varepsilon_0=1582.0\; cm^{-1}$ for two values of $\delta$, $10$ and $50\; meV$ respectively.
\begin{figure}[h]
   \includegraphics[width=0.9\columnwidth]{suppl_images/ando-model}
   \caption{\label{ando-model}Energy and broadening as calculated from the Ando model Eqn.\,(\ref{ando-anomaly}). (a) G-band frequency $\omega_G$. (b) G-band width (FWHM). Black solid lines are calculated for $\delta=10\; meV$ and black dashed lines are calculated for $\delta =50\; meV$. Red solid lines are the asymptotic lines to the G-band frequency with slope $\lambda$ (the electron-phonon coupling) }
\end{figure}
The energy shift (Fig.\,\ref{ando-model}(a)) is characterized by a linear increase of the phonon energy when $\left|E_F(\widetilde{n})\right|$ is changed and is $\propto \lambda E_F$. The asymptotic line with slope $\lambda$ is shown in red. The response function of graphene however has a resonance at $|E_F|=\hbar\omega_G /2$ where electrons are excited precisely to the Fermi edge by a decay of an optical phonon.
As a result the energy shift shows logarithmic divergent softening of the G-band. $E_F=\pm \hbar\omega_G /2$ are marked by vertical lines in Fig.\,\ref{ando-model}. While increasing the broadening $\delta$ diminishes the effect of the divergence, it is still clearly visible for $\delta =50\; meV$. Moreover, strong broadening will change the optical-phonon energy of the neutral system $E_F=0\; eV$ as well, while the linear parts of higher doping will not be affected.

Once the Fermi energy is tuned to $\left| E_F \right| >\hbar\omega_G /2$, the decay of optical phonons into electron hole pairs is prohibited due to Pauli blocking. The lack of the additional decay channel is seen in a decrease of FWHM of the G-band (Fig.\,\ref{ando-model}(b)). The phonon anomaly in graphene has been observed by several experiments \cite{pisana2007breakdown,yan2007electric} which observe a blueshift and change in linewidth of the G-band.

\section{Quantitative Description}
\label{Quant-Descr}
For a quantitative description we describe the gatevoltage dependent energy and FWHM in terms of the Fermi level. Calculation of $E_F$ using Eqn.\,(\ref{CalculationFermiEnergy}) depends on knowledge of the Fermi velocity $v_F$. Fig.\,\ref{FermiVelocityEffect} illustrates the scaling of the data due to different selections of $v_F$.
\begin{figure}[h]
   \includegraphics[width=0.9\columnwidth]{suppl_images/FermiVelocityEffect}
   \caption{\label{FermiVelocityEffect}Effect of changing the Fermi velocity on the position of data points after conversion $V_{bg}\rightarrow E_F$.}
\end{figure}
Selecting a larger $v_F$ places data points at higher energies than lower Fermi velocity values. Using the structure of the phonon response at $B=0\; T$ given by Eqn.\,(\ref{ando-shift}) the Fermi velocity can be estimated. While the theoretically calculated shift and FWHM does not depend on $v_F$, the data has to be scaled by $v_F$ to accomplish a good fit to Eqn.\,(\ref{ando-shift}). 
Selecting any $v_F = \left\lbrace 1.00\times 10^6 \; ms^{-1}, 1.20\times 10^6 \; ms^{-1} \right\rbrace$ can used for acceptable fits to the data. We select $v_F=1.10\times  10^6 \; ms^{-1}$ for best fit results. 

Next we extract the system parameters $\lambda$, $\delta$, $\varepsilon_0$ and $\delta_0$ ($\delta_0$ being the unperturbed phonon linewidth) by fitting of Eqn.\,(\ref{ando-shift}). Initially we set the Fermi velocity $v_F=1.10\times 10^6 \; ms^{-1}$, but explain below how $v_F$ can be found from a gatesweep at $B=0\; T$. Inhomogeneous broadening is included by the following procedure. First we assume the charge distribution to be Gaussian and create a set of normally distributed values of charge carrier density for every gate voltage of interest.
\begin{equation}
\label{GaussianDistro}
f(x) = \frac{1}{\sigma \sqrt{2\pi}}\mathrm{e}^{-\frac{\left(x-\mu\right)^2}{2\sigma^2}}
\end{equation}
The distribution is centered around $\mu = \widetilde{n}(V_{bg})$ given by Eqn.\,(\ref{CalculationDensity}) and has a standard deviation of $\sigma = \delta n$. 

We use the distribution Eqn.\,(\ref{GaussianDistro}) to construct spectral peaks as a superposition of individual spectra of Lorentzian shape. Each value $\widetilde{n}$ in the distribution is converted into an energy by applying Eqn.\,(\ref{CalculationFermiEnergy}) and then used to create the corresponding Lorentzian spectrum where the position and FWHM are given by Eqn.\,(\ref{ando-shift}). 
Finally we apply a Lorentzian fit to the sum signal of the individual peaks to obtain position and FWHM in the presence of spatial charge inhomogeneities.

The outlined procedure for calculation of inhomogeneous broadening is numerically extensive and time intensive. Likewise, the large parameter space makes it difficult to select the optimal choice of fitting values for a global fit, e.g. by $\chi^2$ minimization. Consequently we chose to adjust parameters by eye instead. Initial values were selected in the following way. $\lambda$ and $\varepsilon_0$ follow the slope and position of $\varepsilon_G$ at large Fermi energies where $\Delta \varepsilon_G \propto \lambda E_F$. The difference betweeen FWHM$_{max}$ around $E_F=0\; eV$ FWHM$_{min}$ is determined by $\lambda$ as well (We refer to the FWHM for $|E_F|<\hbar\omega_G /2$ as FWHM$_{max}$ and the FWHM for $|E_F|\gg \hbar\omega_G /2$ as FWHM$_{min}$). Similarly the broadening $\delta$ influences $\varepsilon_G$ and FWHM simultaneously. At $E_F=0\; eV$ the G-band energy upshifts for increasing $\delta$ while the logarithmic divergences at $\hbar\omega_G /2$ become less pronounced. In addition the transition from FWHM$_{max}$ to FWHM$_{min}$ broadens. We select $\delta_0=$FWHM$_{min}$

\begin{figure}[!ht]
   \includegraphics[width=0.9\columnwidth]{suppl_images/ando-fit-ext}
   \caption{\label{ando-fit-ext}G-band position and FWHM and fit results using the Ando model Eqn.\,(\ref{ando-shift}).(a) $B=0\; T$. Black line: no inhomogeneous broadening. Red line: $\delta n = 0.3\times 10^{12}\; cm^{-1}$.(b)$B=0\; T$ for different parameter values (see text). (c) $B=12.6\; T$ with same parameter values as in (b).Thick blue and red lines are the numerical solutions for $\circlearrowleft$ and $\circlearrowright$ transitions including the effects of symmetric transitions. Shaded blue and red regions illustrate shift when not including symmetric transitions}
\end{figure}

We present fitresults in Fig.\,\ref{ando-fit-ext}(a). The fitparameters are: $\lambda = 4.8\times 10^{-3}$, $\delta = 10meV$, $\delta_0=5.5\; cm^{-1}$, $\varepsilon_0 = 1582\; cm^{-1}$, $\delta n = 0.3\times 10^{12}\; cm^{-2}$. We show both numerical results not-including (solid black lines) and results including inhomogeneous broadening (red dashed lines).

Due to the large number of parameters, other choices of their values can be found for acceptable fits to the data. For instance in Fig.\,\ref{ando-fit-ext}(b,c) we chose: $\lambda=4.2\times 10^{-3}$, $\delta=10\; meV$, $\varepsilon_0=1582\; cm^{-1}$, $\delta_0=5.5\; cm^{-1}$ and $v_F=1.18\times 10^6 \; ms^{-1}$.

To describe data in magnetic fields (Fig.\,\ref{ando-fit-ext}(c)) we include the effect of symmetric transitions with coupling strength $\lambda_s=0.2\lambda$ (thick solid lines). While the alternative parameter values describe the energy splitting of the orthogonal phonon channels better than the values initial choice, however the maximum of the FWHM is even further reduced compared to the fit in Fig.\,\ref{ando-fit-ext}(a). A low FWHM$_{max}$ points towards a too small choice of $\lambda$. Also selecting $v_F = 1.18\times 10^6 \; ms^{-1}$ appears as an unnatural large choice given that $v_F$ has been determined to be around $v_F=1.10\times 10^6 \; ms^{-1}$ by measuring the position of magneto-phonon resonances \cite{kossacki2012circular,PhysRevLett.110.227402}. 
Eventually we are able to determine system parameters relevant to electron-phonon interactions in single layer graphene with reasonable confidence, showing the potential of backgate controlled measurements in magnetic fields.

\newpage
\section{Transition weights}
The filling factor dependence of the electron-phonon coupling strength related to a particular LL transition is described by the functions f$_{\mathcal{A},n}$ for both orthogonal circular polarized phonon channels $\mathcal{A}=\circlearrowleft, \circlearrowright$. For convenience we are presenting a graphical representation of f$_{\mathcal{A},n}$ in Fig. \ref{f-factor}.  

\begin{figure}[h!]
   \includegraphics[width=0.9\columnwidth]{suppl_images/f-factor}
   \caption{\label{f-factor}Graphical representation of transition weights f$_{\mathcal{A},n}$ for $\Delta n =\pm 1$ inter (a,c) and intraband (b,d) transitions. (a,b) $\circlearrowright$ polarization, (c,d) $\circlearrowleft$ polarization.}
\end{figure}

\newpage
\section{Relative coupling strength}
We calculated the strength of electron-phonon coupling separately for asymmetric inter and intraband and symmetric transitions ($\lambda_s = 0.2\lambda$). For each transition we select the filling factor that maximizes the transition strength (e.g. by comparing to Fig.\,\ref{f-factor}). 

\begin{center}
\begin{tabular}{|c|c|c|c|c|c|}
\hline
n &$T_n \left(\Delta\varepsilon_G / \lambda T_0 \right)$ & $S_n \left(\Delta\varepsilon_G / \lambda T_0 \right)$ & $T_{-n\rightarrow n}\left(\lambda_s \Delta \varepsilon_G / \lambda T_0\right)$ \\\hline
0 & 2.18 & - & - \\\hline
1 &-0.62 & 0.24 &-0.19 \\\hline
2 &-0.39 & 0.17 & -0.09 \\\hline
3 &-0.31 & 0.15 & -0.06 \\\hline
4 &-0.26 & 0.13 & -0.05 \\\hline
\hline
\end{tabular}
\end{center}


\newpage
\section{Cutoff}

The origin of kinks and splittings in the magnetic field data becomes easily understandable when looking separately at the contributions of individual transitions to the G-band shift. We found that within the range of filling factors we observe, kinks and splitting in the data are related to the lowest lying transitions. For this reason we typically only include the 5 lowest terms in calculating the G-band shift at $B=12.6T$ and find good agreement to our data.    
Theoretically the natural cutoff for transitions has to be selected where the actual dispersion of graphene deviates from the linear approximation around the K and K' points of the Brillouin zone \cite{goerbig2011electronic}. The energy of the LL at the cutoff is approximately given by E$_{N_c} \sim t$ where $t\approx 3eV$ is the nearest neighbor hopping energy. For $B=12.6T$ we find N$_c\approx 450$. Fig.\,\ref{cutoff} shows the effect of including transitions up to N$_c$ using otherwise identical parameters. The effect is an overall downshift of $\sim -1\; cm^{-1}$.  

\begin{figure}[h!]
   \includegraphics[width=0.9\columnwidth]{suppl_images/cutoff}
   \caption{\label{cutoff}G-band splitting in magnetic field $B=12.6 \; T$. Experimental data: grey dots. Numerical simulation (red and blue lines) for high energy cutoff N$_c$=5 (a) and N$_c$=450 (b). Dashed lines are results including effects of symmetric transitions with $\lambda_s=0.2\lambda$.}
\end{figure}



 

% If in two-column mode, this environment will change to single-column format so that long equations can be displayed. 
% Use only when necessary.
%\begin{widetext}
%$$\mbox{put long equation here}$$
%\end{widetext}

% Figures should be put into the text as floats. 
% Use the graphics or graphicx packages (distributed with LaTeX2e).
% See the LaTeX Graphics Companion by Michel Goosens, Sebastian Rahtz, and Frank Mittelbach for examples. 
%
% Here is an example of the general form of a figure:
% Fill in the caption in the braces of the \caption{} command. 
% Put the label that you will use with \ref{} command in the braces of the \label{} command.
%
% \begin{figure}
% \includegraphics{}%
% \caption{\label{}}%
% \end{figure}

% Tables may be be put in the text as floats.
% Here is an example of the general form of a table:
% Fill in the caption in the braces of the \caption{} command. Put the label
% that you will use with \ref{} command in the braces of the \label{} command.
% Insert the column specifiers (l, r, c, d, etc.) in the empty braces of the
% \begin{tabular}{} command.
%
% \begin{table}
% \caption{\label{} }
% \begin{tabular}{}
% \end{tabular}
% \end{table}

% If you have acknowledgments, this puts in the proper section head.
%\begin{acknowledgments}
% Put your acknowledgments here.
%\end{acknowledgments}

% Create the reference section using BibTeX:
\bibliography{magnetophonon}

\end{document}
%
% ****** End of file aiptemplate.tex ******
