\pdfoutput=1

%\documentclass[aip,preprint,graphicx]{revtex4-1}
\documentclass[prl,aps,superscriptaddress,showpacs,reprint]{revtex4-1}
\usepackage{graphicx}
\usepackage{amsmath}
\usepackage{amssymb}



\begin{document}


\title{Electrostatic control of electron-phonon interactions in Graphene in magnetic fields} %Title of paper

\author{Sebastian R\'{e}mi}
\affiliation{Boston University, Department of Physics, 590 Commonwealth Ave, Boston, MA 02215}

\author{Bennett B. Goldberg}

\affiliation{Boston University, Department of Physics, 590 Commonwealth Ave, Boston, MA 02215}
\affiliation{Boston University, Department of Electrical and Computer Engineering}
\affiliation{Boston University, Photonics Center, 8 St. Mary's St, Boston, MA 02215}

\author{Anna K. Swan}
 \email[]{swan@bu.edu}
\affiliation{Boston University, Department of Electrical and Computer Engineering}
\affiliation{Boston University, Department of Physics, 590 Commonwealth Ave, Boston, MA 02215}
\affiliation{Boston University, Photonics Center, 8 St. Mary's St, Boston, MA 02215}

\date{\today}

\begin{abstract}
We investigate the filling factor dependence of the electron-phonon coupling in single layer graphene in magnetic fields by Raman spectroscopy of the G-band optical phonon. At large detuning from magneto-phonon resonances, we observe piecewise linear slopes and splitting of the G-band energy. The splitting is caused by the dichroism of left and right handed circular polarized light due to lifting of the G-band phonon degeneracy by Pauli blocking of inter Landau level transitions. Tuning of the filling factor turns on or off electron-phonon coupling related to transitions further and further away from the nearest magneto-phonon resonance.  We show qualitative and quantitative agreement between data and a linearized model of electron-phonon interactions in magnetic fields.
\end{abstract}




\pacs{81.05.ue,63.22.Rc,73.22.Pr,85.35.-p,71.70.Di}% insert suggested PACS numbers in braces on next line
%81.05.ue: graphene general
%63.22.Rc: Phonons in graphene
%73.22.Pr: Electronic structure of graphene
%85.35.-p: Nanoelectronic devices
%71.70.Di: Landau Levels

\maketitle %\maketitle must follow title, authors, abstract and \pacs

When Dirac fermions in graphene are subjected to a perpendicular magnetic field $B$, the electronic states form discrete, degenerate Landau levels (LL) with energy of $E_{\pm, n}=\pm v_F\sqrt{2e\hbar Bn}$ ($n$ is the Landau level index) \cite{neto2009electronic,goerbig2011electronic}. LL degeneracy is described by  the filling factor $\nu = h\widetilde{n}/eB$, where $\widetilde{n}$ is the surface density of charges \cite{goerbig2007filling}.
The degeneracy of each level is 4, (valley and spin degeneracy). At charge neutrality $\nu = 0$ and the $n=0$ level is half filled. A coherent superposition of LL excitons and phonons can occur via the electron-phonon interaction when the magnetic field is tuned so that a LL transition is in resonance with the phonon energy \cite{ando2007magnetic,goerbig2007filling,goerbig2011electronic}. Systems where the electronic and phonon dephasing is smaller than the electron-phonon interaction strength show pronounced anticrossings of the energy of the phonon and magneto-exciton states \cite{ando2007magnetic,goerbig2007filling,goerbig2011electronic}. The first Raman spectroscopy study of magnetophonon resonance on multi-layer graphene on SiC was not pristine enough to exhibit coherent phonon magneto-exciton states, but demonstrated strong electron-phonon coupling at resonance conditions \cite{faugeras2009tuning} . Coherent magneto-phonon Raman response has since been observed in graphite \cite{PhysRevB.84.235138,kim2012magnetophonon,yan2010observation} and decoupled surface layers of graphene on graphite crystals \cite{kuhne2012polarization,faugeras2011magneto,yan2010observation}. Recently, magneto-phonon resonance has also been observed on single layer graphene exfoliated on SiO$_2$ \cite{PhysRevLett.110.227402,kossacki2012circular}.

Magneto-exciton transitions couple to circularly polarized superpositions, $u^{\circlearrowright}$ and $u^{\circlearrowleft}$, of the degenerate LO and TO modes of the G-band phonon. Symmetry allowed transitions obey the selection rule $\Delta\left|n\right|=\pm 1$ \cite{PhysRevB.84.235138}. Experimentally selective excitation of the orthogonal states is achieved using cross circular polarized optical excitation and detection channels \cite{PhysRevLett.110.227402,kossacki2012circular,kuhne2012polarization}. In a $\sigma^+/\sigma^-$ configuration, only states coupling to $\Delta n=+1$ transitions can be observed, while only states coupling to $\Delta n=-1$ transitions are observed in $\sigma^-/\sigma^+$ polarization configuration.

These selection rules create optical dichroism for doped graphene \cite{goerbig2007filling,ando2007magnetic} and have been observed in graphene on SiO$_2$ due to (partial) Pauli blocking of the initial or final Landau level states of transitions of some  $\Delta n=+1$ and $\Delta n=-1$ transitions \cite{PhysRevLett.110.227402,kossacki2012circular}. While the high quality graphene on SiC or on graphite can be considered charge neutral, exfoliated graphene on SiO$_2$ typically shows accidental doping from sample preparation and impurities in the substrate. So far, only limited control of the filling factor has been achieved using samples that are accidentally doped from the fabrication process \cite{kossacki2012circular}. The accidental doping is changed by annealing or exposure to air and other gases in between measurements of the magneto-phonon resonance by sweeping a magnetic field \cite{PhysRevLett.110.227402}.
For fixed charge carrier density $\widetilde{n}$, the filling factor $\nu$ changes with the magnetic field strength. In order to decouple the effects of magnetic fields and filling factor dependence it would be desirable to tune the number of charges at constant magnetic field by using graphene field effect devices, widely used in transport \cite{novoselov2004electric,novoselov2005two,zhang2005experimental} as well as Raman measurements \cite{pisana2007breakdown,yan2007electric,stampfer2007raman}.

Here we show for the first time charge carrier density dependent magneto Raman measurements on single layer graphene field effect devices at constant magnetic fields. Contrary to previous magneto-phonon studies, we measure the Raman response due to charge tuning in a magnetic field far from magneto-phonon resonance conditions. We observe a pronounced splitting of the G-band as a function of $\nu$, as well as definite changes in slopes of the phonon energy vs LL filling. We show that the structure in the electron-phonon coupling is due to filling (or emptying) of the participating Landau level transitions. The splitting is due to the different filling factor dependent response to left and right hand polarized light. Contrary to on-resonance measurements, no single transition dominates the coupling, and several inter and intra-band transitions have to be considered to account for the experimental observations.
We demonstrate a linear dependence $\propto \nu$ for the non-resonant regime, rather than $\propto \sqrt{\nu}$ as predicted for the on-resonance response.  Our data also suggests an observable influence of symmetric transitions $\Delta n=0$ \cite{kuhne2012polarization}, though not allowed by the magneto-phonon selection rules. We determine the electron-phonon coupling strength in magnetic field  $B=12\: T$ and at $B=0\: T$ and find high level of agreement.

For control of the charge carrier density we fabricated field effect devices based on single layer graphene where the charge density is controlled by the gate voltage. Graphene is exfoliated onto a 300 nm thick SiO$_2$/Si substrate and the devices are fabricated using standard microfabrication processes. We confirm the single layer character by Raman spectroscopy and measurement of the optical contrast \cite{ferrari2006raman,casiraghi2007rayleigh,ni2007graphene,blake2007making}.
Fig.\,\ref{setup}(b) shows an optical image of our sample.

\begin{figure}
\includegraphics{images/setup}
\caption{\label{setup}a) Schematic view of the cryogenic Raman setup using linearly polarized light. b) Optical image of a graphene field effect device. c) G-band Raman spectra at $B=12.6T$. Black spectrum measured at $V_{bg}=-20\: V, \nu=-4.7$. The red spectrum shows visible splitting of the G-band ($V_{bg}=-8\: V,\nu=-1.8$) Solid lines are double-peaked Lorentzian fits}
\end{figure}

Raman measurements are performed at 4K in a He bath cryostat, schematically drawn in Fig.\,\ref{setup}a. The sample is mounted on top of a piezo electric x-y-z stage and placed in the focus of a confocal microscope. Microscope and stage are mounted in a vacuum tube that is filled with low pressure He exchange gas. The tube with the microscope is then inserted in the cryostat in the center of a superconducting magnet. The accessible range of magnetic field is $B=\left\lbrace 0\: T,12.6\: T \right\rbrace$.
The Raman response is excited using a HeNe Laser at $\lambda=632.8\: nm$ with a diffraction limited spot size $\sim 1\: \mu m$. The excitation laser is linearly polarized, although we do not monitor or optimize the polarization. Scattered light is filtered by a long pass filter to remove the laser light, collected by a single mode fiber and analyzed using a conventional grating spectrometer.

Fig.\,\ref{setup}(c) illustrates the effect of applying a backgate voltage $V_{bg}$ at finite magnetic field of $B=12.6\: T$. For $V_{bg}=-20\: V$ the G-band is symmetric with a single peak (black line), but splits into 2 peaks for $V_{bg}=-8\: V$ due to the interaction of the optical phonons with the discrete Landau levels \cite{kias2009} .

Characterization of the phonon response was performed by Raman backgate sweeps in the range $V_{bg}=\left\lbrace -40\: V,40\: V \right\rbrace$ both at $B=0\: T$ as well as $B=12.6\: T$.
\begin{figure}[!hb]
\includegraphics[width=\columnwidth]{images/bcompare}
\caption{\label{bcompare}
Raman intensity map of the G-band Raman spectra as a function of applied backgate voltage. a) Measurements for $B=0\: T$. b) $B=12.6\: T$. A clear splitting is visible for voltages $\left|V\right|\leq 20\: V$. c) and d)  show phonon energies from Lorentzian fits in a) and b). Dashed red line is a fit using a model of the phonon anomaly in graphene \cite{tsuneya2006anomaly}}
\end{figure}

Fig.\,\ref{bcompare} shows a Raman intensity map of the observed spectra as a function of $V_{bg}$. We extract the position of the G-band by fitting the spectra with single ($B=0\: T$) and double lorentzian ($B=12.6\: T$) functions shown in Fig.\,\ref{bcompare}(c) and(d).

Measurements results at $B=0\: T$ are shown in Fig.\,\ref{bcompare}(a) and (c).
The behavior of the G-band  as a function of charge density has previously been studied theoretically and experimentally \cite{pisana2007breakdown,yan2007electric,stampfer2007raman,tsuneya2006anomaly}.  We use the model of \cite{tsuneya2006anomaly} to fit the data, and also include the effects of inhomogeneous broadening due to charge carrier density fluctuations \cite{yan2007electric}. In Fig.\,\ref{bcompare}(c) we plot the fit results (red dashed line) as a function of $V_{bg}$, and extract the following fit values: The electron-phonon coupling strength $\lambda = 4.8\times 10^{-3}$, the phenomenological broadening parameter \cite{tsuneya2006anomaly} $\delta=10\:  meV$, the unperturbed phonon energy at $B=0\: T$, $\varepsilon=1582.0\: cm^{-1}$, the inhomogeneous broadening $\delta \widetilde{n}=0.3\times 10^{12}\: cm^{-2}$ (standard deviation of a Gaussian distribution) and finally the Fermi velocity $v_F=1.10\times 10^6\: ms^{-1}$.  For more details on the qualitative and quantitative description at $B=0\: T$ we refer to the supplementary material.

Fig.\,\ref{bcompare}(b) and (d) shows a very different behavior for $B=12.6\: T$.
G-band splittting starts around $V_{bg}\approx -20\: V$ then reaches a maximum at $V_{bg}\approx-8\: V$, disappears at $V_{bg} = 0\: V$ and repeats symmetrically for $V_{bg}>0\: V$. The largest magnitude of the splitting is $\sim 12\: cm^{-1}$. The nearest magneto-phonon resonances are at the transitions $\left|n\right|=0 \rightarrow \left|n\right|=1$  which occurs at $B=25\: T$ and $\left|n\right|=1 \rightarrow \left|n\right|=2$ at $B=4.1\: T$.

Following \cite{goerbig2007filling,ando2007magnetic,kossacki2012circular}  we consider the  phonon energy $\varepsilon_{\mathcal{A}}$  which is determined from the equation for the poles of the phonon Green’s function, given by:

\begin{eqnarray}
\label{greens-full}
\varepsilon_{\mathcal{A}}^2-\varepsilon_0^2 &=& 2\varepsilon_0 \lambda T_0^2 \left[ \sum_{n=0}^N\left(\frac{f_{\mathcal{A},n}\left(\nu\right) T_n}{\left(\varepsilon_{\mathcal{A}}+i\delta\right)^2-T_n^2}-\frac{1}{T_n}\right)\right.\nonumber \\
& &\left. +\sum_{n=1}^{N}\frac{f_{\mathcal{A},n}\left(\nu\right) S_n}{\left(\varepsilon_{\mathcal{A}}+i\delta\right)^2-S_n^2}\right]
\end{eqnarray}

Here $\varepsilon_0$ is the unperturbed phonon energy, $\lambda$ is the dimensionless electron-phonon coupling parameter and $\delta$ is the phenomenological broadening introduced by Ando \cite{ando2007magnetic}. The two sums are the contribution from all the interband and intraband asymmetric transitions (see Fig.\,\ref{bfield}(a)). The interband transitions have energy $T_n=T_0\left(\sqrt{n+1}+\sqrt{n}\right)$ and intraband transitions $S_n=T_0\left(\sqrt{n+1}-\sqrt{n}\right)$ where $T_0=v_F\sqrt{2e\hbar B}$. For $B=12.6\: T$, $T_0 \approx 140\: meV$ compared with $\sim 196\: meV$ for the G band phonon.
The filling factor dependence of the coupling to the highly degenerate Landau level states is described by the factor $f_{\mathcal{A},n}$. The index $\mathcal{A}$ denotes the two orthogonal circularly polarized phonon states accessed by $\circlearrowleft$- or $\circlearrowright$- circularly polarized light. For interband transitions $f_{\mathcal{A},n}$ is defined by
\begin{eqnarray}
\label{fterm}
f_{\circlearrowright,n}&=&(1+\delta_{n,0})(\bar{\nu}_{-n}-\bar{\nu}_{+(n+1)})\nonumber\\
f_{\circlearrowleft,n}&=&(1+\delta_{n,0})(\bar{\nu}_{-(n+1)}-\bar{\nu}_{+n})
\end{eqnarray}
Here  $\bar{\nu}_{n} $ is the normalized  filling factor describing the fraction of filling of the $n$’th Landau level. It is related to the filling factor $\nu$ by $\bar{\nu}_{\pm n} = \left[\nu -(4(\pm n)-2)\right]/4$ since each Landau level state is fourfold degenerate.
Hence $0<\bar{\nu}_{n}<1$. The definition of $f_{\mathcal{A},n}$ for intraband transition is easily obtained by replacing the index $\mp n$ by $\pm n$.

Eqn.\,(\ref{greens-full}) is valid for all $B$ fields and charge states, although it is cumbersome to use. Near resonance, a single resonant term is dominating, so other terms can be neglected in solving Eqn.\,(\ref{greens-full}). The solution is described by a two level coupled mode model \cite{yan2010observation,PhysRevLett.110.227402}
\begin{equation}
\label{resonant-shift}
\varepsilon_{\mathcal{A}}^{\pm}=\frac{T_n+\varepsilon_0}{2}\pm\sqrt{\left(\frac{T_n-\varepsilon_0}{2}\right)^2+\frac{\lambda T_0^2}{2}f_{\mathcal{A},n}}
\end{equation}
Eqn.\,(\ref{resonant-shift}) describes the anticrossing between the coherent coupled states $\varepsilon_{\mathcal{A}}^{\pm}$. The index $\pm$ refers to the upper and lower branches of the anticrossing.

In the non-resonant regime, where our experiment is performed, the approximation leading to Eqn.\,(\ref{resonant-shift}) is not valid. Since  $\Delta\varepsilon_{\mathcal{A}} = \varepsilon_{\mathcal{A}} - \varepsilon_0$ is small, Eqn.\,(\ref{greens-full}) can be linearized by noting that $\varepsilon_{\mathcal{A}}^2-\varepsilon_0^2 \approx (\varepsilon_{\mathcal{A}}-\varepsilon_0)2\varepsilon_0$, and  replacing the $\varepsilon_{\mathcal{A}}$ in the denominator by the unperturbed phonon frequency $\varepsilon_0$ \cite{ando2007magnetic}. The shift of the G-band in the non-resonant regime is then given by

\begin{eqnarray}
\label{greens}
\Delta\varepsilon_{\mathcal{A}} &=& \mathrm{Re}\left\lbrace \lambda T_0^2 \left[ \sum_{n=0}^N\left(\frac{f_{\mathcal{A},n}\left(\nu\right) T_n}{\left(\varepsilon_{0}+i\delta\right)^2-T_n^2}-\frac{1}{T_n}\right)\right.\right.\nonumber \\
& &\left.\left. +\sum_{n=1}^{N}\frac{f_{\mathcal{A},n}\left(\nu\right) S_n}{\left(\varepsilon_{0}+i\delta\right)^2-S_n^2}\right]\right\rbrace
\end{eqnarray}

The expressions Eqn.\,(\ref{resonant-shift}) and Eqn.\,(\ref{greens}) are distinguished by their filling factor dependence. In the resonance approximation Eqn.\,(\ref{resonant-shift}) the shift is $\propto \sqrt{\nu}$ while in the non-resonant case Eqn.\,(\ref{greens}) is linear in the filling factor $\propto \nu$.
We only include the first 5 terms of Eqn.\,(\ref{greens}) in our calculation, since the neglected terms only cause a small overall downshift of $\sim -1cm^{-1}$ in the range of our measurement (See supplementary material).

We now consider the effect of the charge tuning in Eqn.\,(\ref{greens}) with fixed $B$ field.  In Fig.\,\ref{bfield} (c), the phonon energy is plotted versus filling factor rather than charge density to highlight the correspondence between the filled Landau levels and the extrema of kinks in the slope (orange lines). First we note that since our experiment uses linear polarized light without an analyzer, we measure both  $\sigma^+/\sigma^-$  and $\sigma^-/\sigma^+$ transitions, i.e. both $\varepsilon_{\circlearrowleft}$ and $\varepsilon_{\circlearrowright}$ associated with $\Delta |n|=\pm1$ respectively. The filling factor dependent splitting between the phonon components $\varepsilon_{\mathcal{A}}$ is created by this asymmetry described by the factor $f_{\mathcal{A}}$ in Eqn.\,(\ref{greens}).

\begin{figure*}
\includegraphics{images/fig3_alt}
\caption{\label{bfield} a)Schematic view of the Landau level spectrum at $B=12.6\: T$, filling factor $\nu=2$ and the lowest Landau level transitions participating in magneto-phonon coupling. Filled electronic states are highlighted using orange color. Red and blue arrows show transitions allowed by the selection rule $\Delta\left|n\right|= \pm 1$. Dashed arrows mark Pauli blocked transitions. Circular arrows represent the angular momentum involved in the transitions b) Relative strength and filling factor dependence of individual terms of the phonon self energy. Terms describing interband transitions are shaded red, intraband transitions are shaded green  c) Position of the graphene G-band during a gatesweep at $B=12.6\: T$. Vertical orange lines mark specific filling factors at $\nu=-6,-2,0,2,6$ where the n=-1,0,1 levels are completely filled/depleted with charge carriers ($\nu=0$ - half filling of n=0 level). The calculated magneto-phonon energies according to equation \ref{greens} are plotted as solid red($\Delta n = +1$) and solid blue($\Delta n = -1$) lines. Dashed lines include $\Delta n = 0$ transitions.}
\end{figure*}

For example, in Fig.\,\ref{bfield}(a) we illustrate the level occupation for a filling factor $\nu=2$, i.e. the $n=0$ level is completely filled (partial filling factor $\bar{\nu}_0=1$). Hence,  the transition between $n=-1$ and $ n=0$ is Pauli blocked (dashed lines in figure) and  $f_{\circlearrowleft,0}=0$, while all transitions originating from the n=0 level have maximum strength due to the high density of occupied states that can be excited, with $f_{\circlearrowright,0}=1$.

In order to evaluate the contributions from  the participating inter and intraband transitions, we calculate the individual contribution to the phonon energy shift $\Delta\varepsilon$ from each term in (\ref{greens}) evaluated at $B=12.6\: T$. Shown in Fig.\,\ref{bfield}(b) are the $\Delta\varepsilon$ for the lowest lying inter and intraband transitions for $\circlearrowright$ polarization, i.e. transitions with $\Delta n =+1$. We use the values for $\varepsilon_0$, $\lambda$, $v_F$ and $\delta$ from the $B=0\: T$ fit. Curves are labeled as T$_n$ (red shaded) for interband and S$_n$ (green shaded) for intraband transitions. (T$_n$,f$^\circlearrowright$) is shorthand for a single term $n$ of the summation in Eqn.\,(\ref{greens}) normalized by $\lambda$T$_0$. The largest contribution is due to the (T$_0$,f$^\circlearrowright$) term which shows a strong peak at $\nu=2$. The contribution from the remaining interband transition is strong as well. Shifts due to (T$_1$,f$^\circlearrowright$) are 28\% and (T$_2$,f$^\circlearrowright$) is still at 17.9\% relative to the shift caused by the T$_0$ term.
The action of intraband terms are restricted to a smaller range of $\nu$ values, but their strength can be significant nevertheless (S$_1$/T$_0$ $\approx$ 10\%). The case of the opposite phonon polarization is completely symmetric relative to the point $\nu = 0$.

Finally,  we put these contributions together and compare to our experimental data (Fig.\,\ref{bfield}(c)). The solid red and blue lines are the numerical results for the energies $\varepsilon_\circlearrowright$ and $\varepsilon_\circlearrowleft$.
The splitting between $\varepsilon_\circlearrowright$ and $\varepsilon_\circlearrowleft$ is maximal at $\nu=-2$ and $\nu=2$ where the coupling strength to the T$_0$ transitions corresponding to $\Delta|n|=\pm1$ respectively are strongest. Fig.\,\ref{bfield}(b) also explains the kink in slope $\nu=\pm6$.  The upshift with increasing $|\nu|$  is  caused by the linear decrease of the the T$_1$ transition as the LL $n=\pm1 $ are filled or emptied, respectively. We could not resolve the small splitting of $\approx 1\: cm^{-1}$ between $\varepsilon_\circlearrowright$ and $\varepsilon_\circlearrowleft$ for $\left|\nu\right|>6$ predicted by the model.

The model calculations show a more shallow slope of the low energy magneto-phonon component than the data for $2 < \left|\nu\right| < 6$. A possible cause could be coupling to symmetric transitions $\Delta n = 0$ which are strongly suppressed due to the symmetry of the graphene lattice \cite{PhysRevB.84.235138}. Hybridization of $\Delta n =0$ transition and optical phonons were observed in decoupled graphene layers on natural graphite \cite{faugeras2011magneto,kuhne2012polarization} with a coupling strength of almost 20\% of $\lambda$. We include the effects of the first five $\Delta n = 0$ transitions by adding resonance terms in Eqn.\,(\ref{greens}) with a similarly reduced coupling strength, $\lambda_{symm} =0.2\lambda$. The results are shown as dashed lines in Fig.\,\ref{bfield}(c).

We emphasize that the parameter values determined from fitting to $B=0\: T$ data ($v_F=1.1\times 10^6\: ms^{-1}$, $\lambda=4.8\times 10^{-3}$)  (neglecting charge inhomogeneous broadening), describes the $B=12.6\: T$ data well. In addition, the parameter values agree with those determined in previous experiments. A high Fermi velocity has been observed on graphene on SiO$_2$ ($1.08-1.10\times 10^6\: m/s$) \cite{PhysRevLett.110.227402,kossacki2012circular} while the Fermi velocity for graphene on graphite and SiC layers is in the range  of $1.02-1.04\times 10^6\: m/s$ \cite{faugeras2009tuning,kuhne2012polarization}. It has been suggested that the Fermi velocity depends strongly on screening due to electron-electron interactions \cite{hwang2012fermi}. This could possibly account for the systematic difference in $v_F$ between graphene on different substrates.

To compare $\lambda$ with earlier work, we recall that $\lambda$ is related to the coupling factor $g$ used e.g. in \cite{yan2010observation,PhysRevLett.110.227402} by $g^2=(\lambda T_0^2/2) f_{\mathcal{A},n}$.  We note that  the value of $g$ will depend on both $v_F$ and the magneto-phonon resonance in question via $T_0$, as well as $\widetilde{n}$ via $f_{\mathcal{A},n}$.
The electron-phonon coupling strength $\lambda=4.8\times 10^{-3}$ determined by us,
would result in $g=78.8\: cm^{-1}$  ($ g/\sqrt{B_{0,r}}=15.8 \:  cm^{-1}/T^{1/2}$) at the $T_0$ resonance, and  $ g=32\: cm^{-1}$ ($ g/\sqrt{B_{1,r}}=15.8\: cm^{-1}/T^{1/2}$) at the $T_1$ resonance using $v_F=1.1\times 10^6\: ms^{-1}$ for a charge neutral sample. For graphene on SiO$_2$  in \cite{PhysRevLett.110.227402}, the reported value is nearly identical;  $ g=17\: cm^{-1}/T^{1/2}$ ($\circlearrowright$ polarization, $\widetilde{n}=0.4\times 10^{12}\: cm^{-2}$).    However for the  more pristine sample of  graphene on graphite \cite{yan2010observation},  found $\lambda= 6.2 \times 10^{-3}$, a $23\%$ higher value. Again, the values seems to be affected by the substrate.  For a discussion of the precision of the extracted parameters, we refer to the supplementary material.

To summarize,  we observed a fine structure of the $E_{2g}$ optical phonon in single layer graphene at a magnetic field $B=12.6\: T$ as a function of charge density. We show that the observed behavior is caused by phonon magneto-exciton coupling in the non-resonant regime, which has a linear dependence on filling factor, in contrast to on resonance coupling where the coupling is given by the square root of the filling factor.  By sweeping the charge carrier density, the filling factor dependent coupling strength of orthogonal non-resonant  magneto-phonon states are being turned on and off.  The qualitative behavior of our observations is in good agreement with numerical calculations, and coupling to many Landau level transitions has to be included. The measured  coupling strength, broadening and Fermi velocity is in good agreement with independent observations at $B=0\: T$ and previous experiments.
\\

\begin{acknowledgments}
We thank Mengkun Liu for help with sample preparation and Mark Goerbig and Alex Kitt for discussions.
\end{acknowledgments}


% Create the reference section using BibTeX:
\bibliography{magnetophonon}

\end{document}
%
% ****** End of file aiptemplate.tex ******
