\pdfoutput=1

%\documentclass[aip,preprint,graphicx]{revtex4-1}
\documentclass[prl,aps,superscriptaddress,showpacs,reprint]{revtex4-1}
\usepackage{graphicx}
\usepackage{amsmath}
\usepackage{amssymb}



\begin{document}


\title{Electrostatic control of electron-phonon interactions in Graphene in magnetic fields} %Title of paper

\author{Sebastian R\'{e}mi}
\affiliation{Boston University, Department of Physics, 590 Commonwealth Ave, Boston, MA 02215}

\author{Bennett B. Goldberg}

\affiliation{Boston University, Department of Physics, 590 Commonwealth Ave, Boston, MA 02215}
\affiliation{Boston University, Department of Electrical and Computer Engineering}
\affiliation{Boston University, Photonics Center, 8 St. Mary's St, Boston, MA 02215}

\author{Anna K. Swan}
 \email[]{swan@bu.edu}
\affiliation{Boston University, Department of Electrical and Computer Engineering}
\affiliation{Boston University, Department of Physics, 590 Commonwealth Ave, Boston, MA 02215}
\affiliation{Boston University, Photonics Center, 8 St. Mary's St, Boston, MA 02215}

\date{\today}

\begin{abstract}
We investigate the filling factor dependence of the electron-phonon coupling in single layer graphene in magnetic fields by Raman spectroscopy of the G-band optical phonon. At large detuning from magneto-phonon resonances, we observe a shift and a splitting of the G-band. The splitting is caused by the dichroism of left and right handed circular polarized light due to lifting of the G-band phonon degeneracy by Pauli blocking of inter Landau level transitions. We show that the splitting is proportional to the filling factor characteristic for the non-resonant regime. The tuning of the filling factor turns on or off electron phonon coupling for resonances further and further away from the nearest resonance.  We show qualitative and quantitative agreement between data and numerical models of electron phonon interactions in magnetic fields.
\end{abstract}



\pacs{81.05.ue,63.22.Rc,73.22.Pr,85.35.-p,71.70.Di}% insert suggested PACS numbers in braces on next line
%81.05.ue: graphene general
%63.22.Rc: Phonons in graphene
%73.22.Pr: Electronic structure of graphene
%85.35.-p: Nanoelectronic devices
%71.70.Di: Landau Levels

\maketitle %\maketitle must follow title, authors, abstract and \pacs

When Dirac fermions in graphene are subjected to a perpendicular magnetic field, the electronic states form discrete, degenerate Landau levels (LL) with energy of $E_{\pm, n}=\pm v_F\sqrt{2e\hbar Bn}$ (n is the Landau level index) \cite{neto2009electronic,goerbig2011electronic}. Magneto-excitons are due to electron-phonon interactions which cause  coupled quantum states that are superpositions of optical phonons and magneto-excitons. Magneto-phonon resonance occurs when when the energy difference between Landau levels that satisfy $\Delta\left|n\right|=\pm 1$   is in resonance with the energy of the optical phonon \cite{ando2007magnetic,goerbig2007filling,goerbig2011electronic}. Systems with high coherence where the electronic and phonon dephasing is smaller than the electron phonon interaction strength show pronounced anticrossings of the energy of the phonon magneto-exciton state \cite{ando2007magnetic,goerbig2007filling,goerbig2011electronic}. The first study  of magnetophonon resonance was not pristine enough to exhibit  a coherent magneto phonon state, but demonstrated strong electron phonon coupling at resonance conditions observed by Raman spectroscopy on multi-layer graphene on SiC \cite{faugeras2009tuning} . Coherent magnetophon Raman response has since been observed in graphite \cite{PhysRevB.84.235138,kim2012magnetophonon,PhysRevLett.110.227402,PhysRevB.80.241404,yan2010observation} and decoupled surface layers of graphene on graphite crystals \cite{kuhne2012polarization,faugeras2011magneto,yan2010observation}. Recently magneto-phonon resonance has also been observed on single layer graphene exfoliated on SiO$_2$ \cite{PhysRevLett.110.227402}.  

Magneto-exciton transitions couple to circular polarized superposition $u^{\circlearrowright}$ and $u^{\circlearrowleft}$ of the degenerate LO and TO modes of the G-band phonon. Symmetry allowed transitions obey the selection rule $\Delta\left|n\right|=\pm 1$ in the Landau level index $n$ \cite{PhysRevB.84.235138}. In order to conserve angular momentum, right circular polarized phonon $u^{\circlearrowright}$ couples only to the $-n\rightarrow n+1$ magneto-exciton state and the left circular polarized phonon $u^{\circlearrowleft}$ to the $-(n+1)\rightarrow n$ magnetoexciton state \cite{goerbig2007filling}. Experimentally selective excitation of the orthogonal states is achieved using cross circular polarized excitation and detection channels\cite{PhysRevLett.110.227402,kossacki2012circular}. In a $\sigma^+/\sigma^-$ configuration only states coupling to $\Delta n=+1$ transitions can be observed. Consequently only states coupling to $\Delta n=-1$ transitions are observed in $\sigma^-/\sigma^+$ polarization configuration.

These selection rules create optical dichroism for doped graphene \cite{goerbig2007filling,ando2007magnetic} and  have been observed in graphene on SiO$_2$ due to (partial) Pauli blocking of the initial or final final Landau level states of transitions of some  $\Delta n=+1$ and $\Delta n=-1$ transitions. \cite{PhysRevLett.110.227402,kossacki2012circular} Indeed the coupling strength between optical phonons and magneto-excitons has been predicted to depend on the filling factor $\nu = h\widetilde{n}/eB$ where $\widetilde{n}$ is the surface density of charges \cite{goerbig2007filling}. While the high quality graphene on SiC or on graphite can be considered charge neutral, exfoliated graphene on SiO$_2$ typically shows intrinsic doping from sample preparation and impurities in the substrate. For example the transitions between $n=-1\rightarrow 0$ and $n=0\rightarrow -1$ have equal strength in the neutral system. However if $\nu=2$ the $n=-1\rightarrow 0$ is Pauli blocked due to the complete filling of the n=0 Landau level. Consequently no shift is observed in the $\sigma^+/\sigma^-$ configuration while magneto-phonon resonance is observed in $\sigma^-/\sigma^+$ configuration.

So far only limited tuning and control of the filling factor has been achieved using samples that are either accidentally doped from the fabrication process\cite{,kossacki2012circular}, or where the accidental doping is chnaged by annealing or exposure to air and other gases in between measurements of the magneto-phonon resonance by sweeping a magnetic field \cite{PhysRevLett.110.227402}. However, even for constant charge carrier density $\widetilde{n}$, the filling factor $\nu$ changes with the magnetic field strength because of the change in degeneracy of the Landau levels. In order to decouple the effects of magnetic fields and filling factor dependence it would be desirable to tune the number of charges at constant magnetic field by studying single layer graphene field effect devices, which have been widely used in transport \cite{novoselov2004electric,novoselov2005two,zhang2005experimental} as well as Raman measurements \cite{pisana2007breakdown,yan2007electric,stampfer2007raman}.

Here we show for the first time charge carrier density dependent magneto Raman measurements on single layer graphene field effect devices at constant magnetic fields. Contrary to previous magneto Raman studies, our measurements are not in the magneto-phonon resonance regime. Instead, we measure the Raman response to charge tuning at a magnetic field far from Magneto-phonon resonance conditions, where the signal has mainly phonon character. Most visibly, we observe a pronounced splitting of the G-band as a function of $\nu$ due to the selective activation of coupling to different $\Delta n=+1$ and $\Delta n=-1$ transitions. We show that the splitting is caused by the filling (or emptying) of the participating Landau levels. Contrary to magneto-phonon resonance, no single transition dominates the coupling, and several inter and intra-band transitions have to be considered to account for the experimental observations. We determine the electron-phonon coupling strength in magnetic fields and at $B=0T$ and find high level of agreement between parameters. We demonstrate a linear, rather than quadratic, dependence on  $\sim \nu$ characteristic for the non-resonant regime.  Our data also suggests an observable influence of symmetric transitions $\Delta n=0$ \cite{kuhne2012polarization}, though not allowed by the magneto-phonon selection.

For control of the charge carrier density we fabricated field effect devices based on single layer graphene  where the charge density is controlled by the gate voltage. The graphene is exfoliated onto a 300 nm thick SiO$_2$/Si substrate and the devices are fabricated using standard microfabrication processes.  We confirm the single layer character by Raman spectroscopy and measurement of the optical contrast \cite{ferrari2006raman,casiraghi2007rayleigh,ni2007graphene,blake2007making}. Figure \ref{setup}b shows an optical image of our sample.

\begin{figure}
   \includegraphics{images/setup}
   \caption{\label{setup}a) Schematic view of the cryogenic Raman setup using linearly polarized light. Measurements are performed at $T=4K$ in He exchange gas. b) Optical image of a graphene field effect device. c) G-band Raman spectra at $B=12.6T$. Black spectrum taken at $V_{bg}=-20V, \nu=-4.7$. Red spectrum taken at $V_{bg}=-8V,\nu=-1.8$ shows visible splitting of the G-Band. Solid lines are double-peaked Lorentzian fits}
\end{figure}

Raman measurements are performed at 4K in a He bath cryostat, schematically drawn in Figure \ref{setup}a. The sample is mounted on top of a piezo electric x-y-z stage and placed in the focus of a confocal microscope. Microscope and stage are mounted in a vacuum tube that is filled with low pressure He exchange gas. The tube with the microscope is then inserted in the cryostat such that the sample is located in the center of a superconducting magnet. The accessible range of magnetic field is $B=\left\lbrace 0T,12.6T \right\rbrace$.
Raman spectra are taken using a HeNe Laser at $\lambda$=632.8nm with a diffraction limited spot size $\approx$ 1$\mu m$. The excitation laser is mostly linearly polarized. However, we do not monitor or optimize the polarization. Scattered light is filtered by a long pass filter to remove the laser light, collected by a single mode fiber and analyzed using a conventional grating spectrometer.

Fig \ref{setup}c illustrates the effect of applying a backgate voltage $V_{bg}$ at finite magnetic field of $B=12.6T$. For $V_{bg}=-20V$ the G-Band is symmetric with a single peak (black line), but splits into 2 peaks for $V_{bg}=-8V$ due to the interaction of the optical phonons with the discrete Landau levels. We reported on the observation of this splitting before \cite{kias2009} and present here for the first time a full explanation in terms of the phonon-magnetoexciton coupling.

Characterization of the splitting was performed by Raman backgate sweeps in the range $V_{bg}=\left\lbrace -40V,40V \right\rbrace$ both at $B=0T$ as well as $B=12.6T$.
\begin{figure}[!hb]
   \includegraphics[width=\columnwidth]{images/bcompare}
   \caption{\label{bcompare}Raman intensity map of the G-Band Raman spectra as a function of applied backgate voltage. a) Measurements for $B=0T$. Due to the phonon anomaly in graphene the G-band frequency redshifts towards $V_{bg}=0V$. b) $B=12.6T$. A clear splitting is visible for voltages $\left|V\right|\leq 20V$. Second row shows positions from lorentizan fits. Dashed red line is a fit using a model of the phonon anomaly in graphene}
\end{figure}
Fig. \ref{bcompare} shows a Raman intensity map of the observed spectra as a function of $V_{bg}$. We also extract the position of the G-Band by fitting the spectra with single ($B=0T$) and double lorentzian ($B=12.6T$) functions(lower half of Fig. \ref{bcompare}).

Measurements results at $B=0T$ are shown in Fig. \ref{bcompare}(a).
The behavior of the G-band at vanishing magnetic field is referred to as anomalous behavior of the optical phonon and has been theoretically and experimentally studied \cite{pisana2007breakdown,yan2007electric,stampfer2007raman,tsuneya2006anomaly}. Qualitatively our measurements agree well with previous observations. For large $V_{bg}$ the G-band energy increases and around $V_{bg}=0V$ the Raman spectrum is broadened. 

We adjust the model of \cite{tsuneya2006anomaly} to our data and follow \cite{yan2007electric} to include the effects of inhomogeneous broadening due to charge carrier density fluctuations. For more details on the qualitative and quantitative description at $B=0T$ we refer to the supplementary material. In the lower half of Fig. \ref{bcompare}(a) we plot the fit results (red dashed line) as a function of $V_{bg}$. We specify the following parameters: the electron-phonon coupling strength $\lambda = 4.8\times 10^{-3}$, the phenomenological broadening parameter introduced by Ando \cite{tsuneya2006anomaly} $\delta=10meV$, the unperturbed phonon energy at $B=0T$ $\varepsilon=1582.0cm^{-1}$ and inhomogeneous broadening $\delta \widetilde{n}=0.3\times 10^{12}cm^{-2}$ (standard deviation of a Gaussian distribution). We set the Fermi velocity $v_F=1.10\times 10^6ms^{-1}$.

Figure \ref{bcompare}b) shows the fundamentally different behavior for $B=12.6T$. 
The splitting of the G-band emerges around $V_{bg} \approx -20V$, reaches a maximum at $V_{bg} \approx -8V$ and disappears at $V_{bg} = 0V$ and repeats symmetrically for $V_{bg}>0V$. The largest magnitude of the splitting is $\approx 12cm^{-1}$. The magnetic field strength $B=12.6T$ where the gatesweep was performed is located far away from the nearest magneto-phonon resonances. Resonance with the transition $\left|n\right|=0 \rightarrow \left|n\right|=1$ occurs at $B=25T$ and with $\left|n\right|=1 \rightarrow \left|n\right|=2$ at $B=4.1T$. Lacking resonant enhancement of phonon magneto-exciton coupling we observe significantly less shifting compared to G-Band shifts $>50cm^{-1}$ that have been observed on resonance. 

Since we are observing in the non-resonant regime and observe only small deviations from the unperturbed phonon energy we can conclude that the splitting is not due to anticrossing coupled modes. Instead the splitting is caused by the dichroism due to the filling factor Pauli blocking of Landau level transitions. While $\Delta n=+1$ and $\Delta n=-1$ transitions are selectively excited by probing with orthogonal circular polarized light, we excited both transition channels at the same time using linear polarized light. At $V_{bg}=0V$ the system is neutral ($\nu=0$) and the coupling to both polarized channels is equal so the Raman band is observed at the same energy. For $V_{bg}>0V$ the $n=-1\rightarrow 0$ transition is sequentially turned off due to filling of the $n=0$ Landau level while the $n=0\rightarrow 1$ transition is blocked by depletion the $n=0$ level for $V_{bg}<0V$. 



For a deeper analysis we calculated the energy $\varepsilon_{\mathcal{A}}$ where $\mathcal{A}=\circlearrowright,\circlearrowleft$ of the coupled phonon magneto-exciton states involving $\circlearrowright$- and $\circlearrowleft$- polarized phonons. Following \cite{goerbig2007filling,ando2007magnetic,kossacki2012circular} $\varepsilon_{\mathcal{A}}$ is determined from the equation for the poles of the phonon Green’s function, given by:  

\begin{eqnarray}
\label{greens-full}
\varepsilon_{\mathcal{A}}^2-\varepsilon_0^2 &=& 2\varepsilon_0 \lambda T_0^2 \left[ \sum_{n=0}^N\left(\frac{f_{\mathcal{A},n}\left(\nu\right) T_n}{\left(\varepsilon_{\mathcal{A}}+i\delta\right)^2-T_n^2}-\frac{1}{T_n}\right)\right.\nonumber \\
& &\left. +\sum_{n=1}^{N}\frac{f_{\mathcal{A},n}\left(\nu\right) S_n}{\left(\varepsilon_{\mathcal{A}}+i\delta\right)^2-S_n^2}\right]
\end{eqnarray}

We introduced $\varepsilon_0$ as the unperturbed phonon energy. $\lambda$ is the dimensionless electron phonon coupling parameter and $\delta$ is the phenomenological broadening introduced by Ando\cite{ando2007magnetic}. We include coupling to interband transitions with energy $T_n=T_0\left(\sqrt{n+1}+\sqrt{n}\right)$ and intraband transitions $S_n=T_0\left(\sqrt{n+1}-\sqrt{n}\right)$ where $T_0=v_F\sqrt{2e\hbar B}$ taking account of the $\Delta\left|n\right|=\pm 1$ selection rule. The filling factor dependence of the coupling to the highly degenerate Landau level states is described by the factor $f_{\mathcal{A},n}$. For interband transitions $f_{\mathcal{A},n}$ is defined by
\begin{eqnarray}
\label{fterm}
f_{\circlearrowright,n}&=&(1+\delta_{n,0})(\bar{\nu}_{-n}-\bar{\nu}_{+(n+1)})\nonumber\\
f_{\circlearrowleft,n}&=&(1+\delta_{n,0})(\bar{\nu}_{-(n+1)}-\bar{\nu}_{+n})
\end{eqnarray}
where $0<\bar{\nu}<1$ are partial filling factors describing the fraction of filling of the levels involved in the transition. The definition of $f_{\mathcal{A},n}$ for intraband transition is easily obtained by replacing the index $\mp n$ by $\pm n$.

A schematic view of the magneto-exciton transitions that couple to optical phonons is shown in Fig. \ref{bfield}(a). The first contributing inter- and intraband transitions using red($\Delta n = +1$) and blue arrows($\Delta n = -1$). At any given magnetic field there exist a pair of transitions of same energy (e.g. T$_0$, T$_1$) contributing to one of the magneto-phonon states. However due to the selection rule, the $\Delta n = -1$ transition is always excited from a lower lying energy level than the corresponding $\Delta n = +1$ transition.

The filling factor dependent splitting between the phonon components $\varepsilon_{\pm}$ is created by this asymmetry described by the factor $f_{\pm}$ in (\ref{greens}).
\begin{figure*}
   \includegraphics{images/fig3_alt}
   \caption{\label{bfield} a)Schematic view of the Landau level spectrum at B=12.6T, filling factor $\nu=2$ and the lowest Landau level transitions participating in magneto-phonon coupling. Filled electronic states are highlighted using orange color. Red and blue arrows show transitions allowed by the selection rule $\Delta\left|n\right|= \pm 1$. Dashed arrows mark Pauli blocked transitions. Circular arrows represent the angular momentum involved in the transitions b) Relative strength and filling factor dependence of individual terms of the phonon self energy. Terms describing interband transitions are shaded red, intraband transitions are shaded green  c) Position of the graphene G-band during a gatesweep at $B=12.6T$. The energy of the G-Band was extracted from two-peak Lorentz fits to Raman spectra. Vertical orange lines mark specific filling factors at $\nu=-6,-2,0,2,6$ where the n=-1,0,1 levels are completely filled/depleted with charge carriers ($\nu=0$ - half filling of n=0 level). The calculated magneto-phonon energies according to equation \ref{greens} are plotted as solid red($\Delta n = +1$) and solid blue($\Delta n = -1$) lines. Dashed lines include $\Delta n = 0$ transitions.}
\end{figure*}
For example in Fig. \ref{bfield}(a) we illustrate the level occupation for a filling factor $\nu=2$. Occupied electron states are highlighted using orange color. Here the n=0 level is completely filled (partial filling factor ($\bar{\nu}_0=1$) and there are no available states for transitions into this particular level. Since $f_{-,0}=0$ the transition between n=-1 and n=0 is Pauli blocked (dashed lines in figure). On the other hand all transitions originating from the n=0 level have maximum strength due to the high density of occupied states that can be excited. For example $f_{+,0}=1$ so the magneto-phonon coupling to the transition n=0 to n=+1 reaches its maximum strength. Other transitions are Pauli blocked as well, in particular the intraband transitions are not activated before reaching larger absolute values of $\left|\nu\right|>6$.

In (\ref{greens-full}) all possible Landau level transitions allowed to couple to optical phonons are included. In the magneto-phonon resonance regime one resonant term is dominating so other terms can be neglected in solving (\ref{greens-full}) and the solution reduces to the energy given by a two level coupled mode model \cite{yan2010observation,PhysRevLett.110.227402}
\begin{equation}
\label{resonant-shift}
\varepsilon_{\mathcal{A}}^{\pm}=\frac{T_n+\varepsilon_0}{2}\pm\sqrt{\left(\frac{T_n-\varepsilon_0}{2}\right)^2+\frac{\lambda T_0^2}{2}f_{\mathcal{A},n}}
\end{equation}
Eqn. \ref{resonant-shift} describes the anticrossing between the states $\varepsilon_{\mathcal{A}}^{\pm}$.

In the non-resonant regime, where our experiment is performed, the approximation leading to (\ref{resonant-shift}) is invalid and range of transitions contributes to \ref{greens-full} with same order of magnitude. However as we observe far away from resonance the G-Band shift $\Delta\varepsilon_{\mathcal{A}} = \varepsilon_{\mathcal{A}} - \varepsilon_0$ is small can be approximated by replacing the $\varepsilon_{\mathcal{A}}$ in the denominator of \ref{greens-full} by the unperturbed phonon frequency $\varepsilon_0$ \cite{ando2007magnetic}. The shift of the G-band in the non-resonant regime is then given by

\begin{eqnarray}
\label{greens}
\Delta\varepsilon_{\mathcal{A}} &=& \mathrm{Re}\left\lbrace \lambda T_0^2 \left[ \sum_{n=0}^N\left(\frac{f_{\mathcal{A},n}\left(\nu\right) T_n}{\left(\varepsilon_{0}+i\delta\right)^2-T_n^2}-\frac{1}{T_n}\right)\right.\right.\nonumber \\
& &\left.\left. +\sum_{n=1}^{N}\frac{f_{\mathcal{A},n}\left(\nu\right) S_n}{\left(\varepsilon_{0}+i\delta\right)^2-S_n^2}\right]\right\rbrace
\end{eqnarray}

The expressions (\ref{resonant-shift}) and (\ref{greens}) are significantly distinguished by their filling factor dependence. In the resonance approximation (\ref{resonant-shift}) the shfit is $\sim \sqrt{\nu}$ while in the non-resonant case (\ref{greens}) is linear in the filling factor $\sim \nu$. The linear dependence on is reproduced well in our experimental data as can be seen in Fig. \ref{bcompare}(b) by the linear increase of the energy difference of the splitting.

In order to understand the contributions of the significant inter and intraband transitions, we calculated the individual contributions to the phonon energy shift $\Delta\varepsilon$ based on (\ref{greens}) at $B=12.6T$.
In figure \ref{bfield}(b) we plot the filling factor dependence and relative magnitude of the phonon energy shift due to the lowest lying inter and intraband transitions for $\circlearrowright$ polarization. Without restrictions to generality we only consider transitions with $\Delta n =+1$. The case of the opposite phonon polarization is completely symmetric relative to the point $\nu = 0$. Curves are labeled as T$_n$ (red shaded) for interband and S$_n$ (green shaded) for intraband transitions.  The largest contribution is due to the (T$_0$,f$^\circlearrowright$) term which shows a strong peak at $\nu=2$. In contrast the remaining interband transitions span over several Landau levels.. The relative magnitude of the (T$_0$,f$^\circlearrowright$) term is increased due to the special structure of the n=0 Landau level that decouples the A and B sublattices. It is the origin of the $\delta_{n,0}$ term in (\ref{fterm}). The contribution from the remaining interband transition is strong as well. Shifts due to (T$_1$,f$^\circlearrowright$) are 28\% and (T$_2$,f$^\circlearrowright$) is still at 17.9\% relative to the shift caused by the T$_0$ term.
The action of intraband terms are restricted to a smaller range of $\nu$ values, but their strength can be significant nevertheless (S$_1$/T$_0$ $\approx$ 10\%).

Finally we put these contributions together and compare to our experimental data (Fig. \ref{bfield}(c)). We extracted the G-band position from double lorentzian fits to the experimental spectra. The filling factor $\nu$ is calculated from the backgate voltage using a capacitor model with a dielectric SiO$_2$ layer of thickness d=300nm and constant magnetic field B=12.6T.
The solid red and blue lines are the numerical results for the energies $\varepsilon_\circlearrowright$ and $\varepsilon_\circlearrowleft$.
We want to emphasize that we use exactly the same parameters determined from fitting to $B=0T$ data, neglecting charge inhomogeneous broadening. Moreover the parameters agree well with those determined by previous experiments \cite{PhysRevLett.110.227402,kossacki2012circular}.
For a discussion of the confidence in our parameters we refer to the supplementary material.
For simplicity we only included the first 5 terms of (\ref{greens}) in our calculation, whereas according to \cite{goerbig2011electronic} the sum should extend to the high energy cutoff defined by E$_{N_c} \sim t$ where $t\approx 3eV$ is the nearest neighbor hopping energy. For $B=12.6T$ we find N$_c\approx 450$, however the neglected terms only cause a small overall downshift of $\approx -1cm^{-1}$ in the range of our measurement due to the wide plateau of activation of the corresponding transitions.

The branching and turning points of the measurement data coincide well with the filling factors $\nu=\left\lbrace -6,2,0,2,6 \right\rbrace$ (at vertical orange lines). The splitting between $\varepsilon_\circlearrowright$ and $\varepsilon_\circlearrowleft$ is maximal at $\nu=-2$ and $\nu=2$ where the coupling strength to the T$_0$ transitions is strongest. By comparison to the single term contributions in Fig. \ref{bfield}(b) on can immediately see that the kinks in the data at $\nu=-6$ are caused by the onset of the (T$_0$,f$^\circlearrowleft$) (upshift) and (T$_1$,f$^\circlearrowright$)(downshift) transitions. The lowest intraband transitions are maximal at $\nu=-6$ (S$_1$,f$^\circlearrowleft$) and $\nu=6$ (S$_1$,f$^\circlearrowright$). Their influence is responsible for the energy difference between $\nu=-2$ and $\nu=6$ for $\varepsilon_\circlearrowright$ and $\nu=-6$ and $\nu=2$ for $\varepsilon_\circlearrowleft$. The small splitting of $\approx 1cm^{-1}$ between $\varepsilon_\circlearrowright$ and $\varepsilon_\circlearrowleft$ for $\left|\nu\right|>6$ could not be resolved due to the limited spectral resolution of our experimental setup. 

Our numerical calculations show a more shallow slope of the low energy magneto-phonon component then the data for $2 < \left|\nu\right| < 6$. We believe a likely cause could be coupling to symmetric transitions $\Delta n = 0$ which are strongly suppressed due to the symmetry of the graphene lattice \cite{PhysRevB.84.235138}. Hybridization of $\Delta n =0$ transition and optical phonons were observed in decoupled graphene layers on natural graphite \cite{faugeras2011magneto,kuhne2012polarization} with a coupling strength of almost 20\% of $\lambda$. A potential origin of the coupling might be Landau level mixing due to the interaction with the substrate. Qualitatively the $(n=-1 \rightarrow n=1)$ transitions is activated at the same filling factors $\nu=-6,6$ as the T$_1$ transitions right where the kinks in the data occur (Fig. \ref{bfield}(c)). We included the effects of the first five $\Delta n = 0$ transitions by adding resonance terms in (\ref{greens}). However we replace the coupling strength by $\lambda_{symm} =0.2\lambda$. The results are shown as dashed lines in Fig. \ref{bfield}(c).

To summarize we observed a fine structure of the $E_{2g}$ optical phonon in single layer graphene at a magnetic field B=12.6T. We show that the splitting is caused by phonon magnetoexciton coupling in the non-resonant regime. By changing the charge carrier density we control the effective filling factor dependent coupling strength of the orthogonal magnetophonon states. The qualitative behavior of our observations is in good agreement with numerical calculations and coupling to many Landau level transitions has to be included. We measured the coupling strength, broadening and Fermi velocity in good agreement with independent observations at B=0T and previous experiments.
\\

\begin{acknowledgments}
We would like to thank Mark Goerbig and Alexander Kitt for discussions. We thank Mengkun Liu for help with sample preparation.
\end{acknowledgments}



% If in two-column mode, this environment will change to single-column format so that long equations can be displayed. 
% Use only when necessary.
%\begin{widetext}
%$$\mbox{put long equation here}$$
%\end{widetext}


% \begin{table}
% \caption{\label{} }
% \begin{tabular}{}
% \end{tabular}
% \end{table}

% If you have acknowledgments, this puts in the proper section head.
%\begin{acknowledgments}
% Put your acknowledgments here.
%\end{acknowledgments}

% Create the reference section using BibTeX:
\bibliography{magnetophonon}

\end{document}
%
% ****** End of file aiptemplate.tex ******
