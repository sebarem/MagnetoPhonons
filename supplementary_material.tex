
%\documentclass[aip,preprint,graphicx]{revtex4-1}
\documentclass[prl,aps,superscriptaddress,preprint]{revtex4-1}
\usepackage{graphicx}


\begin{document}



\title{Electrostatic control of magnetophonons in Graphene} %Title of paper

\author{Sebastian R\'{e}mi}
\affiliation{Boston University, Department of Physics, 590 Commonwealth Ave, Boston, MA 02215}

\author{Bennett B. Goldberg}

\affiliation{Boston University, Department of Physics, 590 Commonwealth Ave, Boston, MA 02215}
\affiliation{Boston University, Department of Electrical and Computer Engineering}
\affiliation{Boston University, Photonics Center, 8 St. Mary's St, Boston, MA 02215}

\author{Anna K. Swan}
 \email[]{swan@bu.edu}
\affiliation{Boston University, Department of Electrical and Computer Engineering}
\affiliation{Boston University, Department of Physics, 590 Commonwealth Ave, Boston, MA 02215}
\affiliation{Boston University, Photonics Center, 8 St. Mary's St, Boston, MA 02215}


\date{\today}



\pacs{}% insert suggested PACS numbers in braces on next line

\maketitle %\maketitle must follow title, authors, abstract and \pacs

% Body of paper goes here. Use proper sectioning commands. 
% References should be done using the \cite, \ref, and \label commands
\section{Sample Preparation}
Our field effect samples were prepared by standard microfabrication processes. First we produce single layer graphene by scotch tape exfoliation from cystals of natural graphite. We exfoliate on 300nm thermally grown SiO$_2$ on top of a degenerately doped Si (n-type, $\rho < 0.005 \Omega cm$)wafer \cite{novoselov2005two,novoselov2004electric} and identify single layer graphene samples by optical contrast and Raman measurements. Next we prepare the sample for electron beam lithography by spinning Poly(methyl methacrylate) (to facilitate liftoff we spin a bilayer of PMMA495 followed by PMMA950). Afterwards we define contacts on top of graphene using standard electron beam lithography and evaporate Cr/Au obtaining the sample shown in figure \ref{setup}b. In order to clean the sample from fabrication residues we perform in situ current annealing after cooldown by applying I=1.5mA for $\approx$4h.

\section{Phonon anomaly at B=0T}
\subsection{Raman Gatesweeps at B=0T}
Due to a Kohn anomaly at the $\Gamma$ point of the phonon dispersion the energy of the G-band phonon increases with increasing $\left|E_F(\widetilde{n})\right|$, and a logarithmic divergence when the   $\left|E_F(\widetilde{n})\right|$  equals half the G-band phonon energy \cite{tsuneya2006anomaly,pisana2007breakdown,yan2007electric}, which we clearly observe in our data for $\left|V\right| > 0$. Moreover, in the region $E_F<\hbar \omega_G/2$ phonon decay into electron hole pairs is not Pauli blocked. This additional decay channel broadens the spectral linewidth which explains the fuzzy region around $V\approx 0V$ in the greyscale image, figure \ref{bcompare}a). We describe the energy shift of the G-Band following Ando \cite{tsuneya2006anomaly} where it is given by $\Delta \varepsilon_G=Re(\Pi(\varepsilon_0))$. The self energy $\Pi$ is given by

\begin{equation}
\label{ando-anomaly}
\Pi(\varepsilon_0) = \lambda \varepsilon_F -\frac{\lambda}{4}\left(\varepsilon_0+i\delta\right)\left(\mathrm{ln}\left(\frac{\varepsilon_0+2\varepsilon_F+i\delta}{\varepsilon_0-2\varepsilon_F+i\delta}\right)+i\pi\right)
\end{equation}
where $\lambda$ is the electron-phonon coupling strength and $\delta$ describes broadening effects.

Explain fitting procedure (single double lorentzian)

Better description by double lorentzian fit (second component due to strain? or light pollution in the system)

not able to differenciate around V=0 between fit methods, use single lorentzian data in that region

Probably ok accuracy for G-Band position, less well for width in particular around V=0 expect significant deviations in width
\subsection{Determinatin of $\lambda$, $\delta$ and v$_F$}
Ando model

characteristics of the Ando model, linear in $\lambda$ for large Fermi energies. Start of rise is a rough estimate for v$_F$ but not accurate enough to compete with MPR determined values.

Present curve with reasonable fit parameters

Fit parameters roughly equal to previous MPR experiments

\section{Calculation of non-resonant approximation from Greens funtion}

\section{Transition weights}

\section{relative transition strengths}

table and graphs showing the relative transition strength


 

% If in two-column mode, this environment will change to single-column format so that long equations can be displayed. 
% Use only when necessary.
%\begin{widetext}
%$$\mbox{put long equation here}$$
%\end{widetext}

% Figures should be put into the text as floats. 
% Use the graphics or graphicx packages (distributed with LaTeX2e).
% See the LaTeX Graphics Companion by Michel Goosens, Sebastian Rahtz, and Frank Mittelbach for examples. 
%
% Here is an example of the general form of a figure:
% Fill in the caption in the braces of the \caption{} command. 
% Put the label that you will use with \ref{} command in the braces of the \label{} command.
%
% \begin{figure}
% \includegraphics{}%
% \caption{\label{}}%
% \end{figure}

% Tables may be be put in the text as floats.
% Here is an example of the general form of a table:
% Fill in the caption in the braces of the \caption{} command. Put the label
% that you will use with \ref{} command in the braces of the \label{} command.
% Insert the column specifiers (l, r, c, d, etc.) in the empty braces of the
% \begin{tabular}{} command.
%
% \begin{table}
% \caption{\label{} }
% \begin{tabular}{}
% \end{tabular}
% \end{table}

% If you have acknowledgments, this puts in the proper section head.
%\begin{acknowledgments}
% Put your acknowledgments here.
%\end{acknowledgments}

% Create the reference section using BibTeX:
\bibliography{magnetophonon}

\end{document}
%
% ****** End of file aiptemplate.tex ******
