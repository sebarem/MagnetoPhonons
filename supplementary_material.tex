%% ****** Start of file aiptemplate.tex ****** %
%%
%%   This file is part of the files in the distribution of AIP substyles for REVTeX4.
%%   Version 4.1 of 9 October 2009.
%%
%
% This is a template for producing documents for use with 
% the REVTEX 4.1 document class and the AIP substyles.
% 
% Copy this file to another name and then work on that file.
% That way, you always have this original template file to use.

%\documentclass[aip,preprint,graphicx]{revtex4-1}
\documentclass[prl,aps,superscriptaddress,preprint]{revtex4-1}
\usepackage{graphicx}


\begin{document}

% Use the \preprint command to place your local institutional report number 
% on the title page in preprint mode.
% Multiple \preprint commands are allowed.
%\preprint{}

\title{Supplementary material: Electrostatic control of magnetophonons in Graphene} %Title of paper

% repeat the \author .. \affiliation  etc. as needed
% \email, \thanks, \homepage, \altaffiliation all apply to the current author.
% Explanatory text should go in the []'s, 
% actual e-mail address or url should go in the {}'s for \email and \homepage.
% Please use the appropriate macro for the type of information

% \affiliation command applies to all authors since the last \affiliation command. 
% The \affiliation command should follow the other information.

\author{Sebastian R\'{e}mi}
\affiliation{Boston University, Department of Physics, 590 Commonwealth Ave, Boston, MA 02215}
\author{Anna K. Swan}
\affiliation{Boston University, Department of Electrical and Computer Engineering}
\affiliation{Boston University, Department of Physics, 590 Commonwealth Ave, Boston, MA 02215}
\affiliation{Boston University, Photonics Center, 8 St. Mary's St, Boston, MA 02215}

\author{Bennett B. Goldberg}
 \email[]{goldberg@bu.edu}
\affiliation{Boston University, Department of Physics, 590 Commonwealth Ave, Boston, MA 02215}
\affiliation{Boston University, Department of Electrical and Computer Engineering}
\affiliation{Boston University, Photonics Center, 8 St. Mary's St, Boston, MA 02215}




%\email[]{Your e-mail address}
%\homepage[]{Your web page}
%\thanks{}
%\altaffiliation{}


% Collaboration name, if desired (requires use of superscriptaddress option in \documentclass). 
% \noaffiliation is required (may also be used with the \author command).
%\collaboration{}
%\noaffiliation

\date{\today}



\pacs{}% insert suggested PACS numbers in braces on next line

\maketitle %\maketitle must follow title, authors, abstract and \pacs

% Body of paper goes here. Use proper sectioning commands. 
% References should be done using the \cite, \ref, and \label commands
\section{Sample Preparation}
Our field effect samples were prepared by standard microfabrication processes. First we produce single layer graphene by scotch tape exfoliation from cystals of natural graphite. We exfoliate on 300nm thermally grown SiO$_2$ on top of a degenerately doped Si (n-type, $\rho < 0.005 \Omega cm$)wafer \cite{novoselov2005two,novoselov2004electric} and identify single layer graphene samples by optical contrast and Raman measurements. Next we prepare the sample for electron beam lithography by spinning Poly(methyl methacrylate) (to facilitate liftoff we spin a bilayer of PMMA495 followed by PMMA950). Afterwards we define contacts on top of graphene using standard electron beam lithography and evaporate Cr/Au obtaining the sample shown in figure \ref{setup}b. In order to clean the sample from fabrication residues we perform in situ current annealing after cooldown by applying I=1.5mA for $\approx$4h.
\section{Numerical calculation of magneto-phonon energies}


 The self energy is composed of two parts
\begin{displaymath}
\Pi\left(\omega\right)=\Pi_{inter}+\Pi_{intra} \nonumber
\end{displaymath}
where $\Pi_{inter}$ and $\Pi_{intra}$ are contributions due to interband and intraband transitions respectively. We again follow \cite{ando2007magnetic} to calculate $\Pi$:

\begin{equation}
\Pi^{\pm}_{inter}\left(\omega\right)=\lambda T_0^2\sum_{k=0}{\left(\frac{f^{\pm}_k\left(\nu\right)T_k}{\left(\hbar\omega+i\delta\right)^2-T_k^2}+\frac{1}{T_k}\right)}
\end{equation}
and
\begin{equation}
\Pi^{\pm}_{intra}\left(\omega\right)=\lambda T_0^2\sum_{k=0}{\left(\frac{f^{\pm}_k\left(\nu\right)S_k}{\left(\hbar\omega+i\delta\right)^2-S_k^2}\right)}
\end{equation}

The terms $T_k$ and $S_k$ are the energies of the transitions allowed by the selection rule $\Delta \left| n \right| = \pm 1$. $T_n$ describes the interband transitions $L_{-n(-n-1)}\rightarrow L_{n+1(n)}$ and is given by

\begin{equation}
T_n = T_0\left(\sqrt{n+1}+\sqrt{n}\right), n\geq 0
\end{equation}

where $T_0 = v_F\sqrt{2e\hbar B}$ . $S_n$ describes the intraband transitions $L_{-n-1(n)}\rightarrow L_{-n(n+1)}$ 

\begin{equation}
S_n = T_0\left(\sqrt{n+1}-\sqrt{n}\right)
\end{equation}

% If in two-column mode, this environment will change to single-column format so that long equations can be displayed. 
% Use only when necessary.
%\begin{widetext}
%$$\mbox{put long equation here}$$
%\end{widetext}

% Figures should be put into the text as floats. 
% Use the graphics or graphicx packages (distributed with LaTeX2e).
% See the LaTeX Graphics Companion by Michel Goosens, Sebastian Rahtz, and Frank Mittelbach for examples. 
%
% Here is an example of the general form of a figure:
% Fill in the caption in the braces of the \caption{} command. 
% Put the label that you will use with \ref{} command in the braces of the \label{} command.
%
% \begin{figure}
% \includegraphics{}%
% \caption{\label{}}%
% \end{figure}

% Tables may be be put in the text as floats.
% Here is an example of the general form of a table:
% Fill in the caption in the braces of the \caption{} command. Put the label
% that you will use with \ref{} command in the braces of the \label{} command.
% Insert the column specifiers (l, r, c, d, etc.) in the empty braces of the
% \begin{tabular}{} command.
%
% \begin{table}
% \caption{\label{} }
% \begin{tabular}{}
% \end{tabular}
% \end{table}

% If you have acknowledgments, this puts in the proper section head.
%\begin{acknowledgments}
% Put your acknowledgments here.
%\end{acknowledgments}

% Create the reference section using BibTeX:
\bibliography{magnetophonon}

\end{document}
%
% ****** End of file aiptemplate.tex ******
